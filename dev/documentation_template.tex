% ----------------------------------------
% Documentation for JSLAB
% ----------------------------------------
\documentclass[12pt,a4paper]{article}
\usepackage{tabularx}
\usepackage{geometry}
\usepackage{listings}
\usepackage{fancyhdr}
\usepackage{lastpage}
\usepackage{tikz}
\usepackage{xcolor}
\usepackage{booktabs}
\usepackage{caption}
\usepackage{hyperref}
\usepackage{float}
\usepackage{varwidth}
\usepackage[most]{tcolorbox}

% Define variables
\newcommand{\yearvar}{$JSLAB_PUBLISH_YEAR$}
\newcommand{\appversion}{$JSLAB_APP_VERSION$}

% Colors
\definecolor{jsl-blue}{cmyk}{1,1,0,0}
\definecolor{jsl-orage}{cmyk}{0,.5,1,0}
\definecolor{jsl-green}{cmyk}{1,0,1,0.5}
\definecolor{jsl-gray}{cmyk}{0,0,0,0.5}
\definecolor{jsl-light-gray}{cmyk}{0,0,0,0.1}
\definecolor{jsl-yellow}{RGB}{247,223,30}

% Code
\lstdefinelanguage{JavaScript}{
  morekeywords=[1]{break, continue, delete, else, for, function, if, in,
    new, return, this, typeof, var, void, while, with},
  % Literals, primitive types, and reference types.
  morekeywords=[2]{false, null, true, boolean, number, undefined,
    Array, Boolean, Date, Math, Number, String, Object},
  % Built-ins.
  morekeywords=[3]{eval, parseInt, parseFloat, escape, unescape},
  sensitive,
  morecomment=[s]{/*}{*/},
  morecomment=[l]//,
  morecomment=[s]{/**}{*/}, % JavaDoc style comments
  morestring=[b]',
  morestring=[b]"
}[keywords, comments, strings]

\lstdefinestyle{JavaScriptStyle}{
    language=JavaScript,
    basicstyle=\fontfamily{pcr}\selectfont\footnotesize, 
    commentstyle=\color{jsl-green},
    keywordstyle=\bf\color{jsl-blue},
    stringstyle=\color{jsl-gray},
    breaklines=true,
    showstringspaces=false,
    captionpos=b,
    numbers=left,
    numberstyle=\tiny\color{gray},
    stepnumber=1,
    numbersep=10pt,
    inputencoding=utf8,
    extendedchars=true,
    literate=
        {°}{{\textdegree}}1
        {º}{{\textordmasculine}}1 
        {š}{{\v{s}}}1
        {Š}{{\v{S}}}1
        {č}{{\v{c}}}1
        {Č}{{\v{C}}}1
        {ž}{{\v{z}}}1
        {Ž}{{\v{Z}}}1
        {ć}{{\'{c}}}1
        {Ć}{{\'{C}}}1
        {θ}{{$\theta$}}1
        {µ}{{$\micro$}}1 
}

\newtcbox{\code}[1][]{%
    on line,
    colback=jsl-light-gray,
    colframe=jsl-light-gray, 
    boxrule=0pt, 
    arc=3pt,
    left=2pt,
    right=2pt,
    top=2pt,
    bottom=2pt,
    box align=base,
    fontupper=\ttfamily,
    #1
}

\newtcbox{\codeBlock}[1][]{%
    colback=jsl-light-gray,
    colframe=jsl-light-gray, 
    boxrule=0pt, 
    arc=3pt,
    left=2pt,
    right=2pt,
    top=2pt,
    bottom=2pt,
    box align=base,
    breakable,
    enhanced jigsaw,
    varwidth upper,
    fontupper=\ttfamily,
    #1
}

% Formating
\delimitershortfall=-1pt
\tolerance=1
\emergencystretch=\maxdimen
\hyphenpenalty=10000
\hbadness=10000

% Paper
\geometry{
  paperheight=297mm,
  paperwidth=210mm,
  top=40mm,
  bottom=22mm,
  right=20mm,
  left=25mm,
  headheight=45mm,
  headsep = 10mm,
}

\newcommand{\companyLogo}[2][1]{%
\begin{tikzpicture}[y=1cm, x=1cm, yscale=#1, xscale=#1, every node/.append style={scale=#1}, inner sep=0pt, outer sep=0pt]
  \path[fill=#2,line cap=butt,line join=miter,line width=0.1cm,shift={(0,2.3814000000000006)},scale=0.2646] (48.3133, 7.2898).. controls (48.3133, 7.2898) and (45.5748, 7.2298) .. (42.8363, 7.2385).. controls (42.5511, 7.2394) and (42.2572, 7.2395) .. (41.9563, 7.2387).. controls (37.4414, 7.2255) and (31.3175, 7.1544) .. (28.0006, 6.1414).. controls (21.9344, 4.2889) and (18.5966, 1.6908) .. (15.8504, -1.3797).. controls (15.3536, -1.9429) and (14.706, -2.7974) .. (13.9723, -3.8922).. controls (14.3648, -4.7541) and (14.5615, -5.7655) .. (14.5615, -6.9275).. controls (14.5615, -8.2166) and (14.3276, -9.3004) .. (13.8598, -10.1789).. controls (13.3919, -11.0574) and (12.795, -11.7496) .. (12.0693, -12.2557).. controls (11.3532, -12.7522) and (10.6227, -13.0816) .. (9.8779, -13.2439).. controls (9.5808, -13.3028) and (9.2396, -13.352) .. (8.8643, -13.3936).. controls (7.5711, -16.3864) and (6.3915, -19.737) .. (5.5781, -23.2604) -- (44.9074, -23.2898).. controls (45.292, -19.8927) and (45.6146, -16.9976) .. (45.9133, -14.2994).. controls (45.4314, -13.1253) and (45.19, -11.7645) .. (45.19, -10.2162).. controls (45.19, -7.5942) and (45.8113, -5.5113) .. (47.0529, -3.9666).. controls (47.4282, -0.5661) and (47.8187, 2.9505) .. (48.3133, 7.2898) -- cycle(53.9496, -2.5064).. controls (51.9444, -2.5064) and (50.3266, -3.1673) .. (49.0961, -4.4889).. controls (47.8656, -5.804) and (47.2504, -7.6529) .. (47.2504, -10.0357).. controls (47.2504, -12.2883) and (47.8624, -14.0657) .. (49.0863, -15.3678).. controls (50.3103, -16.6633) and (51.8728, -17.3111) .. (53.7738, -17.3111).. controls (55.3103, -17.3111) and (56.5766, -16.9335) .. (57.5727, -16.1783).. controls (58.5753, -15.4166) and (59.2914, -14.2545) .. (59.7211, -12.692) -- (56.9184, -11.8033).. controls (56.6775, -12.8515) and (56.2803, -13.6197) .. (55.727, -14.108).. controls (55.1736, -14.5963) and (54.5128, -14.8404) .. (53.7445, -14.8404).. controls (52.7029, -14.8404) and (51.8565, -14.4563) .. (51.2055, -13.6881).. controls (50.5544, -12.9199) and (50.2289, -11.6308) .. (50.2289, -9.8209).. controls (50.2289, -8.1152) and (50.5577, -6.8814) .. (51.2152, -6.1197).. controls (51.8793, -5.358) and (52.7419, -4.9771) .. (53.8031, -4.9771).. controls (54.5714, -4.9771) and (55.2224, -5.192) .. (55.7563, -5.6217).. controls (56.2966, -6.0514) and (56.6514, -6.6373) .. (56.8207, -7.3795) -- (59.682, -6.6959).. controls (59.3565, -5.5501) and (58.8682, -4.6712) .. (58.2172, -4.0592).. controls (57.1234, -3.024) and (55.7009, -2.5064) .. (53.9496, -2.5064) -- cycle(1.4217, -2.7506) -- (1.4217, -17.067) -- (4.3123, -17.067) -- (4.3123, -11.6666) -- (6.1971, -11.6666).. controls (7.5057, -11.6666) and (8.505, -11.5982) .. (9.1951, -11.4615).. controls (9.7029, -11.3508) and (10.201, -11.1262) .. (10.6893, -10.7877).. controls (11.1841, -10.4426) and (11.591, -9.9706) .. (11.91, -9.3717).. controls (12.229, -8.7727) and (12.3885, -8.0338) .. (12.3885, -7.1549).. controls (12.3885, -6.0156) and (12.1118, -5.0878) .. (11.5584, -4.3717).. controls (11.005, -3.649) and (10.3182, -3.1803) .. (9.4979, -2.9654).. controls (8.964, -2.8222) and (7.8182, -2.7506) .. (6.0604, -2.7506) -- (1.4217, -2.7506) -- cycle(15.7908, -2.7506) -- (21.8748, -2.7506).. controls (23.4048, -2.7506) and (24.5148, -2.8808) .. (25.2049, -3.1412).. controls (25.9015, -3.3951) and (26.4581, -3.8508) .. (26.8748, -4.5084).. controls (27.2915, -5.1659) and (27.4998, -5.9179) .. (27.4998, -6.7643).. controls (27.4998, -7.8385) and (27.184, -8.7271) .. (26.5525, -9.4303).. controls (25.921, -10.1269) and (24.977, -10.5663) .. (23.7205, -10.7486).. controls (24.3455, -11.1132) and (24.8598, -11.5136) .. (25.2635, -11.9498).. controls (25.6736, -12.386) and (26.2238, -13.1607) .. (26.9139, -14.274) -- (28.6619, -17.067) -- (25.2049, -17.067) -- (23.115, -13.9518).. controls (22.3728, -12.8385) and (21.865, -12.1386) .. (21.5916, -11.8521).. controls (21.3182, -11.5592) and (21.0285, -11.3606) .. (20.7225, -11.2564).. controls (20.4165, -11.1458) and (19.9314, -11.0904) .. (19.2674, -11.0904) -- (18.6814, -11.0904) -- (18.6814, -17.067) -- (15.7908, -17.067) -- (15.7908, -2.7506) -- cycle(32.2953, -2.7506) -- (37.5785, -2.7506).. controls (38.7699, -2.7506) and (39.6781, -2.8417) .. (40.3031, -3.024).. controls (41.143, -3.2714) and (41.8624, -3.7109) .. (42.4613, -4.3424).. controls (43.0603, -4.9739) and (43.516, -5.7486) .. (43.8285, -6.6666).. controls (44.141, -7.5781) and (44.2973, -8.7044) .. (44.2973, -10.0455).. controls (44.2973, -11.2239) and (44.1508, -12.2395) .. (43.8578, -13.0924).. controls (43.4997, -14.134) and (42.9887, -14.9772) .. (42.3246, -15.6217).. controls (41.8233, -16.11) and (41.1462, -16.4908) .. (40.2934, -16.7643).. controls (39.6553, -16.9661) and (38.8025, -17.067) .. (37.7348, -17.067) -- (32.2953, -17.067) -- (32.2953, -2.7506) -- cycle(4.3123, -5.1725) -- (5.7088, -5.1725).. controls (6.7505, -5.1725) and (7.4438, -5.205) .. (7.7889, -5.2701).. controls (8.2576, -5.3548) and (8.645, -5.5663) .. (8.951, -5.9049).. controls (9.257, -6.2434) and (9.41, -6.6731) .. (9.41, -7.1939).. controls (9.41, -7.6171) and (9.2993, -7.9882) .. (9.0779, -8.3072).. controls (8.8631, -8.6262) and (8.5636, -8.8606) .. (8.1795, -9.0104).. controls (7.7954, -9.1601) and (7.0337, -9.235) .. (5.8943, -9.235) -- (4.3123, -9.235) -- (4.3123, -5.1725) -- cycle(18.6814, -5.1725) -- (18.6814, -8.8053) -- (20.8201, -8.8053).. controls (22.2068, -8.8053) and (23.0727, -8.7467) .. (23.4178, -8.6295).. controls (23.7628, -8.5123) and (24.033, -8.3105) .. (24.2283, -8.024).. controls (24.4236, -7.7376) and (24.5213, -7.3795) .. (24.5213, -6.9498).. controls (24.5213, -6.468) and (24.3911, -6.0807) .. (24.1307, -5.7877).. controls (23.8768, -5.4882) and (23.5154, -5.2994) .. (23.0467, -5.2213).. controls (22.8123, -5.1887) and (22.1092, -5.1725) .. (20.9373, -5.1725) -- (18.6814, -5.1725) -- cycle(35.1859, -5.1725) -- (35.1859, -14.6549) -- (37.3441, -14.6549).. controls (38.1514, -14.6549) and (38.7341, -14.6093) .. (39.0922, -14.5182).. controls (39.5609, -14.401) and (39.9483, -14.2024) .. (40.2543, -13.9225).. controls (40.5668, -13.6425) and (40.8207, -13.1835) .. (41.016, -12.5455).. controls (41.2113, -11.901) and (41.309, -11.0253) .. (41.309, -9.9186).. controls (41.309, -8.8118) and (41.2113, -7.9622) .. (41.016, -7.3697).. controls (40.8207, -6.7773) and (40.5473, -6.315) .. (40.1957, -5.983).. controls (39.8441, -5.651) and (39.3982, -5.4264) .. (38.8578, -5.3092).. controls (38.4542, -5.218) and (37.6632, -5.1725) .. (36.4848, -5.1725) -- (35.1859, -5.1725) -- cycle(27.5357, -9.7939) -- (31.1723, -9.7939) -- (31.1723, -11.6123) -- (27.5357, -11.6123) -- (27.5357, -9.7939) -- cycle;
\end{tikzpicture}
}

\newcommand{\jslabLogo}[2][1]{%
\begin{tikzpicture}[y=1cm, x=1cm, yscale=#1, xscale=#1, every node/.append style={scale=#1}, inner sep=0pt, outer sep=0pt]
  \path[fill=jsl-yellow,rounded corners=#2] (0.0, 0.0) rectangle (16.6688, -16.6688);
  \path[fill=black,line width=0.2278cm] (3.5007, -15.6898).. controls (4.6304, -15.6898) and (5.4048, -15.0885) .. (5.4048, -13.7675) -- (5.4048, -9.4127) -- (4.1293, -9.4127) -- (4.1293, -13.7493).. controls (4.1293, -14.387) and (3.8651, -14.551) .. (3.446, -14.551).. controls (3.0087, -14.551) and (2.8265, -14.2503) .. (2.6261, -13.895) -- (1.5875, -14.5237).. controls (1.8881, -15.1614) and (2.4803, -15.6898) .. (3.5007, -15.6898) -- cycle;
  \path[fill=black,line width=0.2278cm] (8.4203, -15.6898).. controls (9.632, -15.6898) and (10.534, -15.0612) .. (10.534, -13.9132).. controls (10.534, -12.8473) and (9.9236, -12.3736) .. (8.8394, -11.9089) -- (8.5206, -11.7723).. controls (7.9739, -11.5354) and (7.7371, -11.3805) .. (7.7371, -10.9979).. controls (7.7371, -10.6881) and (7.9739, -10.4513) .. (8.3475, -10.4513).. controls (8.7119, -10.4513) and (8.9487, -10.6061) .. (9.1674, -10.9979) -- (10.1604, -10.3602).. controls (9.7414, -9.6222) and (9.1583, -9.3398) .. (8.3475, -9.3398).. controls (7.2086, -9.3398) and (6.4798, -10.0686) .. (6.4798, -11.0252).. controls (6.4798, -12.0638) and (7.0902, -12.5558) .. (8.0104, -12.9475) -- (8.3292, -13.0842).. controls (8.9123, -13.3393) and (9.2585, -13.4942) .. (9.2585, -13.9315).. controls (9.2585, -14.2959) and (8.9214, -14.5601) .. (8.393, -14.5601).. controls (7.7644, -14.5601) and (7.4091, -14.2321) .. (7.1358, -13.7857) -- (6.0972, -14.387).. controls (6.4707, -15.1249) and (7.236, -15.6898) .. (8.4203, -15.6898) -- cycle;
  \path[fill=black,line width=0.2278cm] (11.527, -15.6078) -- (15.6085, -15.6078) -- (15.6085, -14.4872) -- (12.8025, -14.4872) -- (12.8025, -9.4127) -- (11.527, -9.4127) -- cycle;
\end{tikzpicture}
}

% Tabele
\newcolumntype{L}[1]{>{\raggedright\let\newline\\\arraybackslash\hspace{0pt}}m{#1}}
\newcolumntype{C}[1]{>{\centering\let\newline\\\arraybackslash\hspace{0pt}}m{#1}}
\newcolumntype{R}[1]{>{\raggedleft\let\newline\\\arraybackslash\hspace{0pt}}m{#1}}

\fancypagestyle{tekst}{
    \fancyhf{}
    \fancyhead[C]{
        \begin{minipage}{16.5cm}
            \begin{minipage}{2cm}
                \jslabLogo[0.1]{0.1cm}
                \vspace{1mm}
            \end{minipage}
            \begin{minipage}{2.6cm}
                \vspace{1mm}
                
                \companyLogo[0.15]{black}
            \end{minipage}
           \begin{minipage}{5cm}
                \footnotesize
                web: \href{https://pr-dc.com}{pr-dc.com} \\
                email: \href{mailto:info@pr-dc.com}{info@pr-dc.com}\\
                github: \href{https://github.com/PR-DC}{github.com/PR-DC}
            \end{minipage}  
           \begin{minipage}{6.5cm}
                \raggedleft {\large\textbf{JSLAB \appversion}} \\
                \raggedleft DOCUMENTATION
            \end{minipage}
            \vspace{-5mm}
        \end{minipage}
    }

    \fancyfoot[C]{\vspace{2mm} \bfseries {\thepage} / \pageref{LastPage}}
    \renewcommand{\headrulewidth}{0.4pt}
    \renewcommand{\headrule}{{\color{jsl-yellow}\hrule}}
    \renewcommand{\footrulewidth}{0.4pt}
    \renewcommand{\footrule}{{\color{jsl-yellow}\hrule}}

}

\makeatletter
\newcommand*{\ov}[1]{%
  $\m@th\overline{\mbox{#1}\raisebox{5mm}{}}$%
}

\renewcommand{\arraystretch}{1.2}
\renewcommand{\aboverulesep}{0}
\renewcommand{\belowrulesep}{0}

% Podesavanja naslova slika
\DeclareCaptionLabelSeparator{bar}{ - }
\captionsetup{labelsep=bar}

% ----------------------------------------
% POCETAK DOKUMENTA
% ----------------------------------------
\begin{document}

% ----------------------------------------
% NASLOVNA STRANA
% ----------------------------------------
\begin{titlepage}

    \centering
    
    \vphantom{vspace}
    
    \vspace{5mm}
    
    \companyLogo[0.5]{black}
	
    \vspace{5mm}
    
    \centerline{\footnotesize{web: \href{https://pr-dc.com}{pr-dc.com}, email: \href{mailto:info@pr-dc.com}{info@pr-dc.com}, github: \href{https://github.com/PR-DC}{github.com/PR-DC}}}
    
    \vspace{5mm}

    \hrule
	
    \vspace{10mm}

    \jslabLogo[0.2]{0.1cm}

    \vspace{8mm}
    
    {\Huge\textbf{JSLAB \appversion} \par}
   
    \vspace{2mm}

    {\LARGE DOCUMENTATION \par}

    \vspace{8mm}
    
   \href{https://github.com/PR-DC/JSLAB}{github.com/PR-DC/JSLAB}
    
    \vspace{8mm}
    
    \hrule
    
    \vfill
    
    \textbf{PRDC d.o.o.}, Novo naselje bb, 22310 Šimanovci, Republic of Serbia

\end{titlepage}

% ----------------------------------------
% Contents
% ----------------------------------------
\pagestyle{tekst}
\tableofcontents
\newpage

% ----------------------------------------
% Start of document
% ----------------------------------------

\section{About JSLAB}

Welcome to the JSLAB Documentation! This guide provides comprehensive information for users and contributors, detailing the features, usage, and contribution guidelines to ensure a consistent and high-quality codebase.

\begin{figure}[H]
    \centering
    \centerline{\jslabLogo[0.3]{0.2cm}}
    \caption{JSLAB logo}
    \label{fig:logo}
\end{figure}

The \textbf{JavaScript Laboratory (JSLAB)} is an open-source environment designed for scientific computing, data visualization, and various other computer operations. Inspired by \textit{GNU Octave} and \textit{Matlab}, JSLAB leverages the advantages of JavaScript, including its blazing speed, extensive examples, backing by some of the largest software companies globally, and the vast community of active programmers and software engineers.

The program was developed to fulfill the need for performing calculations in a programming language that allows for code reuse in later project stages. JavaScript was chosen for its speed, dynamic nature, interpretability, extensive library support, large existing codebase, backing by major software companies, and the ability to create both desktop and mobile applications.

JSLAB offers a streamlined, dual-window interface designed to boost productivity and foster innovation. The main window combines a versatile workspace with a sandbox terminal, allowing users to run, test, and iterate on code in real time. The dedicated editor window introduces the \textbf{.JSL file format}—a plain text format tailored for JSLAB scripts. With advanced linting and intelligent autocompletion, the editor makes it easy to write precise, reusable code with minimal errors.

\subsection{Why Choose JSLAB?}

\begin{itemize}
    \item \textbf{Backed by Leading Investments}

JavaScript is supported by major industry investments, ensuring continuous innovation and robust development. Our commitment to excellence makes JSLAB a trusted choice for professionals and organizations worldwide.

\item \textbf{Powered by JavaScript, Trusted by Giants}

Join the ranks of top companies who leverage JavaScript for their mission-critical applications. With JSLAB, you benefit from the same reliable and scalable technology that powers some of the most advanced projects on Earth and beyond.

\item \textbf{Thriving Community and Massive User Base}

Become part of a vibrant and growing community of JavaScript developter. Extensive support network and active forums ensure you always have the resources and assistance you need to succeed.

\item \textbf{Comprehensive Functionality Comparable to Leading Tools}

JSLAB bridges the gap between JavaScript and specialized scientific tools. Enjoy functionalities equivalent to MATLAB, GNU Octave, Python, R, and Julia, all within a single, unified platform. Perform data analysis, machine learning, numerical computations, and more with ease.

\item \textbf{Seamless and Native GUI with HTML, CSS, and SVG}

Design intuitive and visually appealing graphical user interfaces using native HTML, CSS, and SVG. Create interactive dashboards, custom visualizations, and responsive layouts without the need for additional frameworks.

\item \textbf{Extend with Native Modules via NPM and C++/C}

Enhance JSLAB’s capabilities by integrating native modules from npm, built with C++ and C. Tap into a vast ecosystem of extensions and customize your environment to meet your specific needs, ensuring maximum performance and flexibility.

\item \textbf{Join the JSLAB Revolution Today!}

Experience the seamless integration of powerful scientific computing and the flexibility of JavaScript. Whether you're developing complex algorithms, analyzing vast datasets, or creating innovative applications, JSLAB empowers you to achieve more.

\end{itemize}

\subsection{License}
This program is free software: you can redistribute it and/or modify it under the terms of the GNU Lesser General Public License as published by the Free Software Foundation, either version 3 of the License, or (at your option) any later version.

This program is distributed in the hope that it will be useful, but WITHOUT ANY WARRANTY; without even the implied warranty of MERCHANTABILITY or FITNESS FOR A PARTICULAR PURPOSE. See the GNU Lesser General Public License for more details.

You should have received a copy of the GNU Lesser General Public License along with this program. If not, see \url{https://www.gnu.org/licenses/}.

\vspace{5mm}

Copyright (C) \yearvar\; PR-DC info@pr-dc.com

\vspace{8mm}

\section{About documentation}

This documentation serves as a comprehensive guide for both users and contributors of the JSLAB Library. It covers an introduction to the project, detailed features, installation and setup instructions, user interface overview, practical examples, coding standards, build instructions, contribution guidelines, and mechanisms for providing feedback.

The documentation is structured to provide clear and detailed information, ensuring that both new users and seasoned contributors can effectively utilize and contribute to the JSLAB project.

\section{Coding style}
\label{coding-style}

\subsection{Documentation and Comments}

Clear documentation is essential for maintaining and understanding the codebase. We utilize JSDoc for structured documentation and inline comments to clarify complex logic.

\begin{itemize}
  \item \textbf{JSDoc:} Use JSDoc comments to document files, classes, methods, parameters, and return values.
  
  \item \textbf{Inline Comments:} Add comments to explain non-trivial code segments.
\end{itemize}

\begin{lstlisting}[style=JavaScriptStyle]
/**
* Constructs the JSLAB library environment.
* @param {Object} config Configuration options.
*/
constructor(config) {
  // Initialize properties
  this.config = config;
  // ... other initializations
}
\end{lstlisting}

\subsection{Naming Conventions}
\begin{itemize}
  \item \textbf{Classes and Constants:} Use uppercase letters with underscores (e.g., \code{PRDC\_JSLAB\_LIB}).
  
  \item \textbf{Functions and Methods:} Use camelCase (e.g., \code{getFullFilePath}).
  
  \item \textbf{Variables:} Use lowercase letters with underscores (e.g., \code{new\_file\_path}).
  
  \item \textbf{Meaningful Names:} Choose self-explanatory names that convey the purpose clearly.
\end{itemize}

\subsection{Code Structure and Modularization}
\begin{itemize}
  \item \textbf{Separation of Concerns:} Organize code into separate modules/files based on functionality.
  
  \item \textbf{Object-Oriented Design:} Use ES6 classes to encapsulate related properties and methods.
\end{itemize}

\begin{lstlisting}[style=JavaScriptStyle]
const { PRDC_JSLAB_EVAL } = require('./jslab-eval');

class PRDC_JSLAB_LIB {
  constructor(config) {
  this.eval = new PRDC_JSLAB_EVAL(this);
  // ... other initializations
  }
}
\end{lstlisting}

\subsection{Error Handling}
\begin{itemize}
  \item \textbf{Try-Catch Blocks:} Use try-catch to handle potential errors gracefully.
  
  \item \textbf{Custom Errors:} Throw custom error objects for specific error scenarios.
\end{itemize}

\begin{lstlisting}[style=JavaScriptStyle]
try {
  return this.resolve(file_path);
} catch {
  this.jsl.env.error('Path resolution failed.');
}
\end{lstlisting}

\subsection{State Management and Cleanup}
\begin{itemize}
  \item \textbf{Resource Tracking:} Maintain arrays and objects to track active asynchronous operations.
  
  \item \textbf{Cleanup Methods:} Implement methods to clear resources and reset the environment state.
\end{itemize}

\subsection{Configuration Management}
\begin{itemize}
  \item \textbf{Centralized Configuration:} Use a dedicated class (\code{PRDC\_APP\_CONFIG}) to manage all configuration settings.
  
  \item \textbf{Conditional Configurations:} Adjust settings based on the runtime environment or command-line arguments.
\end{itemize}

\subsection{Best Practices}
\begin{itemize}
  \item \textbf{Consistent Formatting:} Use a consistent code formatter to maintain uniform code style.
  
  \item \textbf{Meaningful Commit Messages:} Write clear and descriptive commit messages that explain the purpose of the changes.
  
  \item \textbf{Modular Code:} Write reusable and modular code to enhance maintainability and scalability.
  
  \item \textbf{Comprehensive Testing:} Implement thorough tests to ensure the reliability of your contributions.
\end{itemize}


\section{Installation}

You can install JSLAB by either downloading the latest stable release from GitHub or by building it from source. Choose the method that best fits your needs.

\subsection{Download the Latest Stable Release}
\begin{itemize}
  \item Visit the JSLAB Releases Page on GitHub Repository: 
  
  \url{https://github.com/PR-DC/JSLAB/releases} 
  
  \item Download the appropriate installer and install program.

  \item Try examples from:

  \url{https://github.com/PR-DC/JSLAB/tree/master/examples}

\end{itemize}

\subsection{Build from Source}

If you prefer to build JSLAB from source, follow the detailed \hyperref[build-instructions]{build instructions} available in this documentation.

\section{Build instructions}
\label{build-instructions}

\subsection{Prerequisites}

In order to download necessary tools, clone the repository, and install dependencies via npm, you need network access.

\begin{itemize}
    \item \textbf{Node.js:} Ensure that Node.js is installed on your system. You can download it from the official website: 
    
    \url{https://nodejs.org/}
    
    \item \textbf{npm:} npm is typically installed alongside Node.js.
    
    \item \textbf{node-gyp:} node-gyp is installed alongside with application but it requires additional tools and libraries depending on your operating system. Follow the instructions for your specific OS from: 
    
    \url{https://github.com/nodejs/node-gyp}
    
    \item \textbf{Git:} Suggested for cloning the repository. Download it from the official website: 
    
    \url{https://git-scm.com/}

\end{itemize}

\subsection{Installation Steps}
\begin{enumerate}
    \item Clone the JSLAB repository:
\begin{verbatim}
   git clone git clone https://github.com/PR-DC/JSLAB.git
\end{verbatim}
    
    \item Navigate to the project directory:
\begin{verbatim}
    cd JSLAB
\end{verbatim}

    \item Install the necessary dependencies:
\begin{verbatim}
    npm install
\end{verbatim}

    \item Start the application:
\begin{verbatim}
    npm start
\end{verbatim}

    \item Check examples from:

    \url{https://github.com/PR-DC/JSLAB/tree/master/examples}

\end{enumerate}

\section{Contributing}

\subsection{Setting Up the Development Environment}

Follow the detailed \hyperref[build-instructions]{build instructions} available in this documentation.

\subsection{Making Changes}

Follow the \hyperref[coding-style]{coding style and best practices} available in this documentation.

\subsection{Submitting Changes}

\begin{enumerate}
    \item Create a new branch for your feature or bugfix:
\begin{verbatim}
    git checkout -b feature/your-feature-name
\end{verbatim}

    \item Make your changes and commit them with clear messages:
\begin{verbatim}
   git commit -m "Add feature X to improve Y"
\end{verbatim}

    \item Push your branch to your forked repository:
\begin{verbatim}
    git push origin feature/your-feature-name
\end{verbatim}

    \item Submit a Pull Request (PR) detailing your changes.
\end{enumerate}

\subsection{Testing}
Before submitting a PR, ensure that all tests pass and add new tests for any new functionality you introduce.

\subsection{Reviewing Process}
All PRs are subject to review by the maintainers. Be prepared to make revisions based on feedback to align with project standards.

\subsection{Best Practices}

\begin{itemize}
    \item \textbf{Consistent Formatting:} Use a consistent code formatter (e.g., Prettier) to maintain uniform code style.

    \item \textbf{Meaningful Commit Messages:} Write clear and descriptive commit messages that explain the purpose of the changes.

    \item \textbf{Modular Code:} Write reusable and modular code to enhance maintainability and scalability.

    \item \textbf{Comprehensive Testing:} Implement thorough tests to ensure the reliability of your contributions.
\end{itemize}

\section{Feedback}

Your feedback is invaluable in improving the JSLAB application. Whether you encounter bugs, have feature requests, or need assistance, please reach out through the following channels:

\begin{itemize}
    \item \textbf{GitHub Issues:} Report bugs or suggest features by opening an issue in the GitHub repository.
    
    \item \textbf{Email}: Contact us directly at \href{mailto:info@pr-dc.com}{info@pr-dc.com} or main author at \href{mailto:mpetrasinovic@pr-dc.com}{mpetrasinovic@pr-dc.com}.
\end{itemize}

We encourage active participation and appreciate all forms of feedback that help us enhance the functionality and usability of JSLAB.

\section{Code references}

$JSLAB_CODE_DATA$

% ----------------------------------------
% End of document
% ----------------------------------------
\end{document}